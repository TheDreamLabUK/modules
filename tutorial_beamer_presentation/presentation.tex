\documentclass{beamer}
\usepackage[utf8]{inputenc}
\usepackage{graphicx} % For including images, if any

% Theme choice
\usetheme{Madrid} % A common, clean theme
% Or try others like: Berlin, Boadilla, CambridgeUS, Copenhagen, Darmstadt, Dresden, Frankfurt, Goettingen, Hannover, Ilmenau, Juanlespins, Luebeck, Malmoe, Marburg, Montpellier, PaloAlto, Pittsburgh, Rochester, Singapore, Szeged, Warsaw

\title[Creative Tech Guide]{A Creative Technologist's Guide to GitHub & AI-Powered Workflows}
\subtitle{Tutorial Summary}
\author{Roo Code Assistant (Generated)}
\date{\today}

\institute{Derived from the Workshop Tutorial}

\begin{document}

% Title Page
\begin{frame}
  \titlepage
\end{frame}

% Table of Contents
\begin{frame}
  \frametitle{Outline}
  \tableofcontents
\end{frame}

% --- Content will be added below ---

% Placeholder for Chapter 0
\section{Introduction & Workshop Overview}
\begin{frame}
  \frametitle{Chapter 0: Introduction & Workshop Overview}
  \begin{itemize}
    \item Welcome to the comprehensive guide for creative technologists.
    \item Focus: GitHub, Git, VS Code, and AI-driven coding assistants.
    \item Goal: Establish a robust, cost-free, and powerful toolchain.
    \item Benefits: Version control, collaboration, free web hosting (GitHub Pages), AI assistance.
  \end{itemize}
\end{frame}

\begin{frame}
  \frametitle{Original Workshop Agenda Highlights}
  \begin{itemize}
    \item Understanding Git & Version Control
    \item Accounts & Installations (GitHub, Git, VS Code, API Keys)
    \item Core Git Workflow (Clone, Edit, Commit, Push)
    \item Collaboration, Recovery, GitHub Pages
    \item AI Helpers (Roo Code)
  \end{itemize}
\end{frame}

\begin{frame}
  \frametitle{The Rationale: A Powerful Ecosystem}
  \begin{itemize}
    \item \textbf{GitHub:} Centralised hub for code, portfolios, collaboration, and free hosting.
    \item \textbf{VS Code:} Versatile code editor with seamless Git/GitHub integration.
    \item \textbf{AI (Roo Code with Gemini):} Code generation, explanation, refactoring, vibe coding, persistent context.
    \item Collectively: Manage complex projects, foster collaboration, leverage AI with minimal cost.
  \end{itemize}
\end{frame}

% Placeholder for Chapter 1
\section{Why Git? Understanding Version Control}
\begin{frame}
  \frametitle{Chapter 1: Why Git? Understanding Version Control}
  \begin{itemize}
    \item Git addresses challenges in complex projects:
    \begin{itemize}
        \item Safe experimentation (branches).
        \item Tracking and reverting changes (commits).
        \item Efficient collaboration.
        \item Showcasing work.
    \end{itemize}
  \end{itemize}
\end{frame}

\begin{frame}
  \frametitle{Core Git Concepts}
  \begin{itemize}
    \item \textbf{Repository (Repo):} Directory tracking all changes.
    \item \textbf{Commit:} Snapshot of your project at a point in time.
    \item \textbf{Branch:} Parallel version for independent development.
    \item \textbf{Remote:} Cloud-hosted copy (e.g., on GitHub).
  \end{itemize}
  \vfill
  \textit{Diagram: Local Computer \(\leftrightarrow\) Staging Area \(\leftrightarrow\) Local Repo \(\leftrightarrow\) Remote Repo (GitHub). Branches diverge and merge from Local Repo.}
\end{frame}

\begin{frame}
  \frametitle{Git vs. GitHub}
  \begin{itemize}
    \item \textbf{Git:} The underlying version control software (local, offline capable).
    \item \textbf{GitHub:} Web-based platform hosting Git repositories with additional collaboration tools.
    \item Analogy: Git is the engine, GitHub is a popular car model built around that engine.
  \end{itemize}
\end{frame}

% --- Further chapters will be added here ---

\section{Essential Setup: Accounts & Installations}
\begin{frame}
  \frametitle{Chapter 2: Essential Setup}
  \begin{itemize}
    \item One-time process to establish your digital toolkit.
    \item Provides a robust, free, and powerful environment.
    \item Key steps:
    \begin{enumerate}
        \item Create GitHub Account
        \item Install Git \& VS Code
        \item Set up Google Cloud API Key (for AI)
        \item Configure Roo Code in VS Code
    \end{enumerate}
  \end{itemize}
  \vfill
  \textit{Diagram: Setup workflow from GitHub account to Roo Code config.}
\end{frame}

\subsection{GitHub Account}
\begin{frame}
  \frametitle{2a: Creating Your GitHub Account}
  \begin{itemize}
    \item Go to \texttt{github.com/signup}.
    \item Provide email, create password, choose username (professional \& memorable).
    \item Verify email address.
    \item Select the \textbf{Free} plan.
    \item Importance: Professional visibility, community engagement, identity.
  \end{itemize}
\end{frame}

\subsection{Git \& VS Code Installation}
\begin{frame}
  \frametitle{2b: Installing Git \& VS Code}
  \textbf{Installing Git:}
  \begin{itemize}
    \item macOS: Check with `git --version`; install/update via Homebrew (`brew install git`).
    \item Windows: Download from `gitforwindows.org`; use default options.
    \item Verify: `git --version`.
  \end{itemize}
  \pause
  \textbf{Initial Git Configuration:}
  \begin{block}{Terminal Commands}
    \texttt{git config --global user.name "Your Name"}\\
    \texttt{git config --global user.email "your@email.com"}\\
    \texttt{git config --global init.defaultBranch main}
  \end{block}
\end{frame}

\begin{frame}
  \frametitle{2b: Installing Git \& VS Code (Continued)}
  \textbf{Installing VS Code:}
  \begin{itemize}
    \item macOS: Homebrew (`brew install --cask visual-studio-code`) or download from `code.visualstudio.com`.
    \item Windows: Download from `code.visualstudio.com`; use default options.
  \end{itemize}
  \pause
  \textbf{Recommended VS Code Extensions:}
  \begin{itemize}
    \item \textbf{Roo Code (AI Assistant):} For AI-powered coding.
    \item \textbf{Git Graph:} Visualise Git history.
    \item \textbf{Markdown Preview Mermaid Support:} For diagrams in Markdown.
  \end{itemize}
\end{frame}

\subsection{Google Cloud API Key for AI}
\begin{frame}
  \frametitle{2c: Setting up Google Cloud API Key}
  \begin{itemize}
    \item Needed for AI features in Roo Code (Google Gemini models).
    \item Google Cloud Free Tier: Often includes free credits (e.g., \$300 for 90 days). Payment info required for activation.
    \item Steps:
    \begin{enumerate}
        \item Sign in to Google Cloud Console.
        \item Create/Select a Project.
        \item Navigate to APIs \& Services > Credentials.
        \item Create API Key (copy and store securely).
        \item Enable "Gemini API" in the Library.
        \item Restrict API Key (recommended for security).
    \end{enumerate}
  \end{itemize}
\end{frame}

\subsection{Configuring Roo Code}
\begin{frame}
  \frametitle{2d: Configuring Roo Code in VS Code}
  \begin{itemize}
    \item Open Roo Code panel (kangaroo icon). Click settings (cogwheel).
    \item \textbf{Profile Configuration (for Google Gemini):}
    \begin{itemize}
        \item Profile Name (e.g., "Gemini Pro Rate Limited").
        \item API Provider: Google Gemini.
        \item Paste your API Key.
        \item Select AI Model (e.g., "Gemini 2.5 Pro Preview").
        \item Set Rate Limit (e.g., 30000ms) to manage costs.
        \item Save.
    \end{itemize}
    \item \textbf{Modes:} Ask, Code (for generation/modification), Architect.
    \item \textbf{Auto-Approve:} Enable Read/Write for file operations.
    \item Test with a simple prompt (e.g., "Explain Git").
  \end{itemize}
\end{frame}

\section{The Core Git Workflow}
\begin{frame}
  \frametitle{Chapter 3: The Core Git Workflow}
  \begin{itemize}
    \item Introduces fundamental Git operations.
    \item Cloning repositories, essential commands, basic version control tasks.
  \end{itemize}
\end{frame}

\subsection{Cloning a Repository}
\begin{frame}
  \frametitle{3.1 Cloning a Repository}
  \begin{itemize}
    \item Creates a local copy of an existing repository (all files \& history).
    \item \textbf{Steps:}
    \begin{enumerate}
        \item Find Repository URL on GitHub (\texttt{<> Code} button).
        \item Clone via VS Code (Recommended: "Clone Git Repository...").
        \item Alternative: Command line (`git clone <url>`).
    \end{enumerate}
  \end{itemize}
\end{frame}

\subsection{Essential Git Commands}
\begin{frame}
  \frametitle{3.2 Essential Git Commands}
  \begin{columns}[T] % Align tops
    \begin{column}{.5\textwidth}
      \begin{itemize}
        \item \texttt{git status}: Show changes.
        \item \texttt{git add <file> | .}: Stage changes.
        \item \texttt{git commit -m "msg"}: Commit staged changes.
        \item \texttt{git log}: Display history.
      \end{itemize}
    \end{column}
    \begin{column}{.5\textwidth}
      \begin{itemize}
        \item \texttt{git pull}: Fetch \& merge remote changes.
        \item \texttt{git push}: Upload local commits.
        \item \texttt{git branch}: List/create/delete branches.
        \item \texttt{git checkout -b <name>}: Create \& switch branch.
        \item \texttt{git merge <branch>}: Merge branch.
      \end{itemize}
    \end{column}
  \end{columns}
  \vfill
  \textit{Diagram: Working Directory \(\xrightarrow{add}\) Staging Area \(\xrightarrow{commit}\) Local Repo \(\xrightarrow{push}\) Remote Repo. Remote \(\xrightarrow{pull}\) Local.}
\end{frame}

\subsection{Guided Exercise Summary}
\begin{frame}
  \frametitle{3.3 Guided Exercise: Your First Git Project}
  \begin{itemize}
    \item Create a project folder.
    \item Initialise Git repository (`git init` or via VS Code).
    \item Create/modify a file (e.g., `README.md`).
    \item Stage (`git add`) and commit (`git commit`) changes.
    \item Create a new branch (`git checkout -b feature-branch`).
    \item Make changes, stage, and commit on the new branch.
    \item Switch back to `main` (`git checkout main`).
    \item Merge the feature branch (`git merge feature-branch`).
    \item View history (e.g., using Git Graph extension).
  \end{itemize}
\end{frame}

\section{Collaboration \& Recovery}
\begin{frame}
  \frametitle{Chapter 4: Collaboration \& Recovery}
  \begin{itemize}
    \item Explores effective collaboration using GitHub.
    \item Covers methods for recovering from common mistakes.
  \end{itemize}
\end{frame}

\subsection{Forks vs. Branches}
\begin{frame}
  \frametitle{4.1 Forks vs. Branches}
  \begin{itemize}
    \item \textbf{Branches (Private Team Work):}
    \begin{itemize}
        \item Collaborators have direct write access to the same repository.
        \item Work on features in separate branches within the shared repo.
        \item Integrate changes via Pull Requests (PRs) within the same repo.
    \end{itemize}
    \item \textbf{Forks (Public/Open-Source Contributions):}
    \begin{itemize}
        \item For projects you don't have write access to.
        \item A "fork" is your personal copy of another's repository.
        \item Make changes in your fork, then open a PR to the original (upstream) repo.
    \end{itemize}
  \end{itemize}
  \vfill
  \textit{Diagrams: Branching model (shared repo) vs. Forking model (contributing to external repo).}
\end{frame}

\subsection{Opening a Pull Request (PR)}
\begin{frame}
  \frametitle{4.2 Opening a Pull Request (PR)}
  \begin{itemize}
    \item A formal proposal to merge changes from your branch/fork into another (often `main`).
    \item Central to code review and collaboration on GitHub.
    \item \textbf{Typical Workflow:}
    \begin{enumerate}
        \item Push your feature branch to GitHub (`git push -u origin <branch-name>`).
        \item Create PR on GitHub: Compare \& pull request, select base/compare branches, add title/description.
        \item Review \& Discussion: Team members comment, suggest changes. Iterate with more commits if needed.
        \item Merge PR: After approval, merge changes into the base branch. Delete feature branch (good practice).
    \end{enumerate}
    \item VS Code extensions (e.g., "GitHub Pull Requests and Issues") allow managing PRs in-editor.
  \end{itemize}
\end{frame}

\subsection{Undo Recipes}
\begin{frame}[fragile] % fragile for verbatim code
  \frametitle{4.3 Undo Recipes: Recovering from Mistakes}
  \begin{itemize}
    \item \textbf{Undo last commit (keep changes):} \texttt{git reset --soft HEAD\textasciitilde1}
    \item \textbf{Discard unstaged edits (specific file):} \texttt{git checkout -- <file>}
    \item \textbf{Discard all unstaged edits:} \texttt{git checkout .} or \texttt{git restore .}
    \item \textbf{Unstage a file:} \texttt{git reset HEAD <file>} or \texttt{git restore --staged <file>}
    \item \textbf{Amend last commit message:} \texttt{git commit --amend -m "New msg"} (Avoid on pushed commits)
    \item \textbf{Revert a pushed commit (new commit):} \texttt{git revert <commit\_sha>} (Safer for shared history)
    \item \textbf{Recover deleted branch (find SHA with \texttt{git reflog}):} \texttt{git branch <new\_name> <sha>}
  \end{itemize}
\end{frame}

\begin{frame}
    \frametitle{Undo Recipes: Important Considerations}
    \begin{itemize}
        \item \textbf{\texttt{git reset} vs. \texttt{git revert}:}
        \begin{itemize}
            \item \texttt{git reset} (especially \texttt{--hard}) rewrites history. Safe for local, unpushed commits. \textbf{Avoid on pushed/shared history.}
            \item \texttt{git revert} creates a new commit that undoes previous changes. Safer for shared history.
        \end{itemize}
        \item Many undo operations are available via VS Code's Source Control panel UI.
    \end{itemize}
\end{frame}

\section{GitHub Pages: Your Free Portfolio Site}
\begin{frame}
  \frametitle{Chapter 5: GitHub Pages}
  \begin{itemize}
    \item Host static websites directly from GitHub repositories for free.
    \item Ideal for portfolios, project showcases, documentation.
    \item Serves HTML, CSS, JS (and Markdown converted to HTML).
    \item URL Structures:
    \begin{itemize}
        \item User/Org Site: \texttt{<username>.github.io} (from repo named \texttt{<username>.github.io})
        \item Project Site: \texttt{<username>.github.io/<repository-name>}
    \end{itemize}
  \end{itemize}
\end{frame}

\subsection{Setting Up GitHub Pages}
\begin{frame}
  \frametitle{5.2 Setting Up GitHub Pages for a Repository}
  \begin{enumerate}
    \item \textbf{Prepare Files:} Website needs at least an \texttt{index.html}. Often placed in root or \texttt{/docs} folder.
    \item \textbf{Push to GitHub:} Commit and push your website files.
    \item \textbf{Configure Settings:} In repo `Settings` > `Pages`.
    \item \textbf{Choose Source:} "Deploy from a branch". Select branch (e.g., `main`) and folder (e.g., `/(root)` or `/docs`). Save.
    \item \textbf{Wait for Deployment:} Site will be live at the provided URL.
  \end{enumerate}
\end{frame}

\subsection{Custom Domains}
\begin{frame}
  \frametitle{5.4 Using Custom Domains}
  \begin{itemize}
    \item Use your own domain (e.g., \texttt{www.yourproject.com}).
    \item \textbf{Steps:}
    \begin{enumerate}
        \item Add custom domain in GitHub Pages settings.
        \item Configure DNS records with your domain registrar:
        \begin{itemize}
            \item \textbf{Apex domain (\texttt{yourdomain.com}):} Four `A` records pointing to GitHub IPs (e.g., 185.199.108.153) OR `ALIAS`/`ANAME` record to \texttt{username.github.io}.
            \item \textbf{Subdomain (\texttt{www.yourdomain.com}):} `CNAME` record pointing to \texttt{username.github.io}.
        \end{itemize}
        \item Wait for DNS propagation (can take up to 48 hours).
        \item Verify and Enforce HTTPS in GitHub settings.
    \end{itemize}
    \item Roo Code can help explain DNS concepts and troubleshoot.
  \end{itemize}
\end{frame}

\subsection{Advanced: Private Repo to Public GH Pages}
\begin{frame}
  \frametitle{5.5 Advanced: Publishing from Private Repo}
  \begin{itemize}
    \item Keep source code private, publish built static site to a public repo's GitHub Pages.
    \item Uses GitHub Actions.
    \item \textbf{Overview:}
    \begin{enumerate}
        \item Private source repo, public hosting repo.
        \item GitHub Action in private repo: builds site, pushes to public repo's `gh-pages` branch.
        \item Public repo serves Pages from `gh-pages`.
    \end{enumerate}
  \end{itemize}
\end{frame}

\begin{frame}
  \frametitle{Private to Public GH Pages: Setup Steps}
  \begin{enumerate}
    \item \textbf{Create Personal Access Token (PAT):} With `repo` and `workflow` scopes. Store securely.
    \item \textbf{Add PAT as Secret in Private Repo:} E.g., `DEPLOY_TOKEN`.
    \item \textbf{Create GitHub Actions Workflow File} (e.g., \texttt{.github/workflows/deploy.yml}):
    \begin{itemize}
        \item Checks out code.
        \item Sets up build environment (e.g., Node.js, Hugo).
        \item Runs build commands.
        \item Uses an action like \texttt{peaceiris/actions-gh-pages} to deploy to `external_repository` (public repo) using the PAT. Specify `publish_dir` and `publish_branch`.
    \end{itemize}
    \item \textbf{Enable GitHub Pages in Public Repo:} Serve from the deployment branch (e.g., `gh-pages`).
  \end{enumerate}
  \textit{Roo Code can help generate workflow YAML, explain syntax, and troubleshoot.}
\end{frame}

\section{AI-Powered Workflows with Roo Code}
\begin{frame}
  \frametitle{Chapter 6: AI-Powered Workflows with Roo Code}
  \begin{itemize}
    \item Integrating Git/GitHub \& VS Code with AI (Roo Code extension + Gemini API).
    \item Creates an AI-powered coding assistant directly in the editor.
  \end{itemize}
\end{frame}

\subsection{What is Roo Code?}
\begin{frame}
  \frametitle{6.1 What is Roo Code? (Recap)}
  \begin{itemize}
    \item VS Code extension acting as an AI co-pilot.
    \item Capabilities:
    \begin{itemize}
        \item Generate, explain, refactor, debug code.
        \item Answer questions, generate documentation.
    \end{itemize}
    \item Uses configured API key (e.g., Google Gemini).
    \item Connects AI to your file system with checks and balances.
  \end{itemize}
\end{frame}

\subsection{Using Roo Code: "Vibe Coding"}
\begin{frame}
  \frametitle{6.2 How to Use Roo Code: "Vibe Coding"}
  \begin{itemize}
    \item Iterative, exploratory, conversational approach.
    \item \textbf{Process:}
    \begin{enumerate}
        \item Open Roo Code sidebar (kangaroo icon).
        \item Interact via chat: Be specific, conversational, iterate.
        \item Leverage editor context: Use code selection, active file/project.
        \item Apply suggestions, manage output (copy, or direct write if permitted).
        \item \textbf{Crucially, use Git for version control with AI changes.}
        \item Control Roo Code Modes (Ask, Code, Architect).
        \item Manage rate limiting \& costs (e.g., 30s limit for free tier).
    \end{enumerate}
  \end{itemize}
\end{frame}

\subsection{Prompt Ideas \& Local File Access}
\begin{frame}
  \frametitle{6.3 Prompt Ideas \& 6.4 Local File Access}
  \textbf{Prompt Ideas for Creative Technologists:}
  \begin{itemize}
    \item Project scaffolding (e.g., "Lay out a Python eye tracker project...").
    \item Documentation (e.g., "Refactor README to beginner tutorial.").
    \item Creative asset generation/modification (e.g., "Suggest color-blind safe palette...").
    \item Complex document/diagram creation (e.g., Gantt charts with Mermaid).
    \item Understanding existing code.
  \end{itemize}
  \pause
  \textbf{Power of AI with Local File Access:}
  \begin{itemize}
    \item Superior contextual understanding (reads multiple project files).
    \item Bulk operations (generate directory structures, refactor across files).
    \item Persistent memory within a session.
  \end{itemize}
  \textit{This local integration + Git = dynamic and powerful development.}
\end{frame}

\section{Advanced Typesetting: LaTeX with WSL2, TeXLive, and VS Code}
\begin{frame}
  \frametitle{Chapter 8: LaTeX with WSL2, TeXLive, \& VS Code}
  \begin{itemize}
    \item For high-quality documents (complex formulae, structured layouts).
    \item Setup: WSL2 (Ubuntu), TeXLive, VS Code + LaTeX Workshop extension.
    \item Goal: Seamless LaTeX development in a familiar coding environment.
  \end{itemize}
\end{frame}

\subsection{WSL2 \& TeXLive Setup}
\begin{frame}
  \frametitle{8.1-8.3: WSL2 \& TeXLive Setup}
  \textbf{Setting up WSL2 \& Ubuntu:}
  \begin{itemize}
    \item Install Ubuntu from Microsoft Store (e.g., Ubuntu 22.04 LTS).
    \item Install Windows Terminal (recommended).
    \item Basic Linux commands: `cd`, `ls`, `mkdir`, `sudo apt update/upgrade`.
  \end{itemize}
  \pause
  \textbf{Installing TeXLive on Ubuntu:}
  \begin{itemize}
    \item Update package lists: `sudo apt update`.
    \item Install TeXLive: `sudo apt install texlive texlive-latex-extra`.
    \item (Note: Installs Ubuntu repo version, generally stable).
    \item Explore CTAN (\texttt{ctan.org}) for packages/documentation.
  \end{itemize}
\end{frame}

\subsection{VS Code \& LaTeX Workshop}
\begin{frame}[fragile]
  \frametitle{8.4-8.5: VS Code \& LaTeX Workshop}
  \textbf{Connecting VS Code to WSL2:}
  \begin{itemize}
    \item Install "WSL" extension (by Microsoft) in VS Code.
    \item Connect to WSL (Cmd/Ctrl+Shift+P > "WSL: Connect to WSL").
    \item Open project folder within WSL environment.
  \end{itemize}
  \pause
  \textbf{Setting up LaTeX Workshop Extension:}
  \begin{itemize}
    \item Install "LaTeX Workshop" (by James Yu) in WSL-connected VS Code.
    \item Configure build tools in workspace \texttt{.vscode/settings.json} for `pdflatex`.
    \item Key settings: `latex-workshop.latex.tools` and `latex-workshop.latex.recipes`.
    \item Features: Auto-compile on save, PDF viewer, SyncTeX.
    \item Consider "Code Spell Checker" extension.
  \end{itemize}
\end{frame}

\subsection{Roo Code for LaTeX \& Manual Compilation}
\begin{frame}
  \frametitle{8.6-8.7: Roo Code for LaTeX \& Manual Compilation}
  \textbf{Enhancing LaTeX Workflow with Roo Code:}
  \begin{itemize}
    \item Understand complex LaTeX syntax.
    \item Generate LaTeX structures (articles, beamer slides).
    \item Debug compilation errors.
    \item Find relevant LaTeX packages.
    \item Refine content.
    \item \textit{Use Git for version control with AI-generated LaTeX.}
  \end{itemize}
  \pause
  \textbf{Manual Compilation (Understanding):}
  \begin{itemize}
    \item Create simple \texttt{.tex} file.
    \item Compile from WSL terminal: \texttt{pdflatex mydocument.tex}.
    \item Confirms TeXLive setup and basic process.
  \end{itemize}
\end{frame}

\section{Reference Cheat Sheet}
\begin{frame}[fragile]
  \frametitle{Chapter 9: Reference Cheat Sheet (Git Commands)}
  \begin{itemize}
    \item \textbf{Configuration (Once):}
    \begin{itemize}
        \item \texttt{git config --global user.name "Name"}
        \item \texttt{git config --global user.email "email@example.com"}
        \item \texttt{git config --global init.defaultBranch main}
    \end{itemize}
    \pause
    \item \textbf{Starting Project:}
    \begin{itemize}
        \item \texttt{mkdir project; cd project; git init}
    \end{itemize}
    \pause
    \item \textbf{Connecting to Remote:}
    \begin{itemize}
        \item \texttt{git remote add origin <url>}
        \item \texttt{git push -u origin main}
    \end{itemize}
  \end{itemize}
\end{frame}

\begin{frame}[fragile]
  \frametitle{Cheat Sheet: Everyday Workflow}
  \begin{itemize}
    \item \texttt{git status} - Check changes
    \item \texttt{git add .} or \texttt{git add <file>} - Stage changes
    \item \texttt{git commit -m "message"} - Commit staged
    \item \texttt{git push} - Push to remote
    \item \texttt{git pull} - Pull from remote
  \end{itemize}
  \pause
  \textbf{Branching:}
  \begin{itemize}
    \item \texttt{git branch} - List branches
    \item \texttt{git branch <name>} - Create branch
    \item \texttt{git checkout <name>} - Switch branch
    \item \texttt{git checkout -b <name>} - Create \& switch
    \item \texttt{git merge <name>} - Merge branch
    \item \texttt{git branch -d <name>} - Delete merged branch
  \end{itemize}
\end{frame}

\begin{frame}[fragile]
  \frametitle{Cheat Sheet: Viewing History \& Undoing}
  \textbf{Viewing History:}
  \begin{itemize}
    \item \texttt{git log} - Basic history
    \item \texttt{git log --oneline --graph --decorate --all} - Compact graph
  \end{itemize}
  \pause
  \textbf{Undoing Things (Use with care):}
  \begin{itemize}
    \item Unstage: \texttt{git reset HEAD <file>} or \texttt{git restore --staged <file>}
    \item Discard working dir changes: \texttt{git checkout -- <file>} or \texttt{git restore <file>}
    \item Amend last commit: \texttt{git commit --amend -m "new message"}
  \end{itemize}
  \textit{Refer to official Git documentation for more.}
\end{frame}

\section{Conclusion}
\begin{frame}
  \frametitle{Conclusion & Next Steps}
  \begin{itemize}
    \item This tutorial provides a foundation for a modern creative technology workflow.
    \item Key tools: Git, GitHub, VS Code, and AI (Roo Code).
    \item Practice these concepts to enhance your projects and collaboration.
    \item Explore further: Advanced Git, specific AI prompts, deeper LaTeX integration.
  \end{itemize}
  \vfill
  \centering
  Thank you for following this guide!
\end{frame}

% How Roo Code can help with Beamer presentations
\begin{frame}
  \frametitle{AI Assistance for This Presentation}
  \begin{itemize}
    \item This Beamer presentation structure and initial content were generated with AI assistance (like Roo Code).
    \item Roo Code can help:
    \begin{itemize}
        \item Summarise text for slides.
        \item Suggest Beamer code structures.
        \item Explain LaTeX or Beamer syntax.
        \item Refine slide content for clarity and conciseness.
        \item Generate draft content for new sections based on prompts.
    \end{itemize}
    \item Remember to review and refine AI-generated content for accuracy and style.
  \end{itemize}
\end{frame}

\end{document}